\documentclass[pdf, 10pt, aspectratio=169, bigger, unicode]{beamer}


\usepackage[utf8]{inputenc}
\usepackage[T1, T2A]{fontenc}
\usepackage[english,russian]{babel}
\usepackage{pscyr}

\usepackage{graphics}
\usepackage{graphicx}
\usepackage{graphpap}
\usepackage{ragged2e}
\usepackage{amsfonts}
\usepackage{amsmath}
\usepackage{amssymb}
\usepackage{amstext}
\usepackage{indentfirst}
\usepackage{amsthm}
\usepackage{hyperref}
\usepackage{tikz}
\usepackage{pgf}
\usepackage{pgfplots}
\usepackage{multicol}
\usepackage{slashbox}
\usepackage{tabularx}
\usepackage{bibunits}
\usepackage{epstopdf}
\usepackage{algorithmicx}

 
\usepackage{float}
\usepackage[ruled]{algorithm}
\usepackage{algpseudocode}



\usefonttheme[onlymath]{serif}
\usetikzlibrary{decorations}
\usetikzlibrary{arrows,automata}
\usetikzlibrary{positioning}

\usetikzlibrary{arrows,shapes,snakes,automata,backgrounds,petri}
%Нужно включать, если используется "тема" (стиль оформления) по умолчанию
\usepackage{beamerthemesplit}
\usetheme{Warsaw}


\usecolortheme{seahorse}


%\captionsetup[ruled]{labelsep=period}

\renewcommand{\thealgorithm}{\arabic{algorithm}}
\floatname{algorithm}{Алгоритм}
\algrenewcommand\algorithmicrequire{\textbf{Вход: }}
\algrenewcommand\algorithmicensure{\textbf{Выход: }}
\algrenewcommand\algorithmicwhile{\textbf{До тех пока}}
	\algrenewcommand\algorithmicdo{\textbf{выполнять}}
	\algrenewcommand\algorithmicrepeat{\textbf{Повторять}}
	\algrenewcommand\algorithmicuntil{\textbf{Пока выполняется}}
	\algrenewcommand\algorithmicend{\textbf{Конец}}
	\algrenewcommand\algorithmicif{\textbf{Если}}
	\algrenewcommand\algorithmicelse{\textbf{иначе}}
	\algrenewcommand\algorithmicthen{\textbf{тогда}}
	\algrenewcommand\algorithmicfor{\textbf{Для}}
	\algrenewcommand\algorithmicforall{\textbf{Выполнить для всех}}
	\algrenewcommand\algorithmicfunction{\textbf{Функция}}
	\algrenewcommand\algorithmicprocedure{\textbf{Процедура}}
	\algrenewcommand\algorithmicloop{\textbf{Зациклить}}
	\algrenewcommand\algorithmicreturn{\textbf{Возвратить}}
	\algrenewtext{EndWhile}{\textbf{Конец цикла}}
	\algrenewtext{EndLoop}{\textbf{Конец зацикливания}}
	\algrenewtext{EndFor}{\textbf{Конец цикла}}
	\algrenewtext{EndFunction}{\textbf{Конец функции}}
	\algrenewtext{EndProcedure}{\textbf{Конец процедуры}}
	\algrenewtext{EndIf}{\textbf{Конец условия}}
	\algrenewtext{EndFor}{\textbf{Конец цикла}}
	\algrenewtext{BeginAlgorithm}{\textbf{Начало алгоритма}}
	\algrenewtext{EndAlgorithm}{\textbf{Конец алгоритма}}
	\algrenewtext{BeginBlock}{\textbf{Начало блока. }}
	\algrenewtext{EndBlock}{\textbf{Конец блока}}
	\algrenewtext{ElsIf}{\textbf{иначе если }}
		
\algloop{description}
	\algnewcommand\algorithmicdescription{\textbf{Описание алгоритма}}


\mode<presentation>
{
    \usefonttheme[onlymath]{serif}
		%\usetheme{Copenhagen}
        %\usetheme{Warsaw}
        %\usetheme{Darmstadt}
        %\usetheme{Frankfurt}
        %\usetheme{AnnArbor}
        %\usetheme{CambridgeUS}
    \setbeamercovered{transparent}	
	\setbeamertemplate{footline}
{%
 \leavevmode%
    \hbox{%
        \begin{beamercolorbox}[wd=.5\paperwidth,ht=2.5ex,dp=1.125ex,leftskip=.3cm,rightskip=.3cm]{author in head/foot}%
            \usebeamerfont{author in head/foot}%\insertframenumber{}/\inserttotalframenumber%
                        \hfill\insertshortauthor
        \end{beamercolorbox}%
        \begin{beamercolorbox}[wd=.5\paperwidth,ht=2.5ex,dp=1.125ex,leftskip=.3cm,rightskip=.3cm plus1fil]{title in head/foot}%
            \usebeamerfont{title in head/foot}\insertshorttitle
						\hfill \insertframenumber{}/\inserttotalframenumber%
        \end{beamercolorbox}}%
    \vskip0pt%
}
\setbeamertemplate{navigation symbols}{} 
\setbeamertemplate{navigation symbols}{}
}

\bibliographystyle{unsrt}
\makeatletter
\addto\captionsrussian{
    \renewcommand{\figurename}{Рисунок}
    %\def\figurename{Рисунок}
}


%Более крупный шрифт для подзаголовков титульного листа
\setbeamerfont{institute}{size=\normalsize}

%Задание команды (\bluetext) для выделения конкретным (синим) цветом
%(используйте \alert для выделения цветом выбранной "темы")
\setbeamercolor{bluetext_color}{fg=blue}
\newcommand{\bluetext}[1]{{\usebeamercolor[fg]{bluetext_color}#1}}

\renewcommand{\raggedright}{\leftskip=0pt \rightskip=0pt plus 0cm}

\renewcommand{\rmdefault}{ftm}

%Если используется последовательное появление пунктов списков на слайде
%(не злоупотребляйте в слайдах для защиты дипломной работы), чтобы
%еще непоявившиеся пункты были все-таки немножко видны.
\setbeamercovered{transparent}

\title[ЗАДАЧА MARL  ДЛЯ СЕТИ СВЕТОФОРОВ]{ЗАДАЧА MARL ДЛЯ СЕТИ СВЕТОФОРОВ}
\author[Тисленко Т.И.]{ \bf Тисленко Тимофей Иванович}
\institute[ИМиФИ СФУ]{	{\footnotesize ФГАОУ ВО <<СИБИРСКИЙ ФЕДЕРАЛЬНЫЙ УНИВЕРСИТЕТ>>\\

    Институт математики  и фундаментальной информатики\\[-2pt]
		
	%Кафедра высшей и прикладной математики
	}

 % \vspace{0.5cm}
%{\footnotesize Направление \ 01.037.02 Прикладная математика }

  \vspace{0.2cm}
    {\footnotesize Научный руководитель --- к.ф.-м.н.,  доцент   {\sf Д.В. Семенова }					}
 }
\date[\today]{\footnotesize Томск, ITMM	 2021}

\begin{document}

%%%%%%%%%%%%%%%%%%%%%%%%%%%%%%%%%%%%%%%%
\begin{frame}
\titlepage

 \end{frame}

\normalsize
%%%%%%%%%%%%%%%%%%%%%%%%%%%%%%%%%%

\begin{frame}
	\frametitle{Актуальность}
    \justifying
    В Красноярске стремительно с каждым годом растет количество желающих стать автомобилистом.
    По данным агентства «Автостат» на 1 января 2020 Красноярский край оказался на 12 месте в топ-20 регионов по объему автомобильного парка в России.
    С количеством автомобилистов увеличивается и время, которое проводится в пробках.
\end{frame}
%%%%%%%%%%%%%%%%%%%%%%%%%%%%%%%%%%

\begin{frame}
    \frametitle{Существующие модели адаптивных систем светофоров}
    \justifying
    \begin{figure}[h!]
		\centering
            
			\includegraphics[scale = 0.6]{images/issues.png}
		    \caption{1. Модели адаптивных систем светофоров}
            \label{fig:issues}
	\end{figure}


\end{frame}
%%%%%%%%%%%%%%%%%%%%%%%%%%%%%%%%%%

\begin{frame}
    \frametitle{Цели и задачи}
    \justifying
    \begin{block}{Цель работы}
        \justifying
        Разработка и исследование математической модели мультиагентной системы для задачи оптимизации движения на перекрестке.
    \end{block}

    %\pause

    \begin{block}{Задачи}
        \begin{enumerate} \justifying
            \item Сделать обзор литературы по соответствующей тематике.
            \item Построить математическую модель MARL.
            \item Разработать алгоритм решения задачи MARL.
            \item Провести вычислительные эксперименты.
        \end{enumerate}
    \end{block}

\end{frame}

%%%%%%%%%%%%%%%%%%%%%%%%%%%%%%%%%%%%%%%%%%%%%%%%%%%%%%%%%
\begin{frame}
    \justifying
    \frametitle{Основные определения}
    \begin{block}{Определения} 
        \begin{enumerate} \justifying
            \item	\textbf{Интеллектуальным агентом} называется метаобъект, наделенный долей субъектности,
            взаимодействующий с другими агентами и средой, выполняющий определенные функции для достижения поставленных целей.
            \item	\textbf{Средой} называется множество объектов, не принадлежащих агенту.
            \item 	\textbf{Задачами/ресурсами} называются объекты, распределяемые агентами в ходе достижения целей.
            \item 	\textbf{Мультиагентная система} – совокупность взаимосвязанных агентов.
            \item 	\textbf{RL(Reinforcement Learning)} — обучение с подкреплением, где
            в роли учителя выступает среда. 
        \end{enumerate}
    \end{block}
					


\end{frame}

%%%%%%%%%%%%%%%%%%%%%%%%%%%%%%%%%%%%%%%%%%%%%%%%%%%%%%%%%
\begin{frame}
\justifying
\frametitle{Постановка задачи}


    \begin{block}\justifying
        \begin{enumerate}[-] \justifying    
            \item Модель среды – марковская однородная цепь с конечным числом действий $\textbf{A}$ и состояний $\textbf{S}$ и дискретным временем $t$;
            \item cреда --- участок дорожной сети, где рассчитывается время проезда машин через перекрестки; отсчет времени начинается за 100м до стоп-линий светофоров;
            \item множество агентов $K$ --- все светофоры находящиеся в на участке дорожной сети;
            \item задержка  --- суммарное засеченое время машин, проходящих через отрезки дороги, в момент времени $t$;
            \item $p_{s s'}$ --- вероятность того, что система из состояния $s$ при выборе решения $a$ попадает в состояние $s'$, полностью определяется состоянием, в которое переходит процесс. 
            \item фазы светофора меняются последовательно;
        \end{enumerate}
    \end{block}




\end{frame}

%%%%%%%%%%%%%%%%%%%%%%%%%%%%%%%%%%%%%%%%%%%%%%%%%%%%%%%%%
\begin{frame}
\justifying
\frametitle{Постановка задачи}

    \begin{block}\justifying
        \begin{enumerate}[-] \justifying    
            \item пространство состояний агента $k$ $S^k$  --- фазы светофора, определяются классом вычетов по модулю $n^k=|S^k|$: $S^k = n^k\mathbb{Z}  = \left\{s^{(0)} = 0, s^{(1)} = 1, \ldots, s^{(n^k-1)} = n^k-1\right\}$,  где  $s^{(i)}$ интерпретируется как <<активна фаза $i$>>;
            \item Множеством решений определяется характеристикой кольца $n^k\mathbb{Z}$, т.е. не превыщает ее,
            $A =\left\{a^{(0)} = 0, a^{(1)} = 1, \ldots, a^{(n^k-1)} = n^k-1 \right\}$, где $a^{(j)}$ интерпретируется как 
        <<сменить фазу $j$ раз>>. 
            
        \end{enumerate}
    \end{block}

    


    \begin{figure}[h!]
        \centering
        \begin{tikzpicture}[->,>=stealth',shorten >=1pt,auto,node distance=3cm,
        semithick]
    \tikzstyle{every state}=[draw,text=black]
    
    \node[state] (A)  {$s^{(0)}$};
    \node[state] (B) [right of=A] {$s^{(1)}$};
    
    
    \path (A) edge[bend right] node {$p_{01}$} (B);
    \path (B) edge[bend right] node {$p_{10}$} (A);
    \path (A) edge[loop left, dashed] node {$p_{00}$} (A);
    \path (B) edge[loop right, dashed] node {$p_{11}$} (B);
    \end{tikzpicture}
        \caption{Стохастический граф управляемого процесса смены фаз для двухфазного светофора.  Пунктиром обозначено действие $a^{(0)}$, а сплошной --- $a^{(1)}$.
        }
        \label{fig:stohgraph}
    \end{figure}

\end{frame}

%%%%%%%%%%%%%%%%%%%%%%%%%%%%%%%%%%%%%%%%%%%%%%%%%%%%%%%%%
\begin{frame}
\justifying
\frametitle{Постановка задачи}


\begin{block}\justifying
    \begin{enumerate}[-] \justifying    
        \item совокупное состояние среды $\textbf{s}_t$ в момент времени $t$ \linebreak $\textbf{s}_t = \{s_t^1, s_t^2, \ldots, s_t^K\}\in S^1\times S^2\times\ldots\times S^K$;
        \item совокупное управление $\textbf{a}_t$ в момент времени $t$ $\textbf{a}_t = \{a_t^1, a_t^2, \ldots, a_t^K\} \in A^1\times A^2\times\ldots\times A^K$; 
        \item множество соседних агентов $N(k)$ для светофора $k$ --- агенты, с которыми взаимодействует $k$;
        \item для пары агентов $k$ и $j\in N(k)$ множество их совместных действий $\textbf{a}^{kj} \in A^k\times A^j$ и совместных состояний $\textbf{s}^{kj} \in S^k\times S^j$;
        \item под задержкой $r(s_t, a_t; s_{t+1})$ на фазе $s_t$  будем понимать суммарное засечённое время всех машин, проходящих через отрезок дороги. 
        
    \end{enumerate}
\end{block}


\end{frame}

%%%%%%%%%%%%%%%%%%%%%%%%%%%%%%%%%%%%%%%%%%%%%%%%%%%%%%%%%
\begin{frame}
    \justifying
    \frametitle{Постановка задачи}
    
    
    \begin{figure}[h!]
        %\centering
        %\includegraphics[width =0.2\textwidth]{images/intersection2.png}
        \begin{tikzpicture}[->,>=stealth',shorten >=1pt,auto,node distance=3.8cm,
          semithick, scale=0.1]
      \tikzstyle{every state}=[draw,text=black]
      
      \node[state] (s0)  {$s_{0}$};
      \node[state] (s1) [right of=s0] {$s_{1}$};
      \node[state] (s2) [below  of=s0] {$s_{2}$};
      \node[state] (s3) [below  of=s1] {$s_{3}$};
      
      \path (s0) edge[sloped,midway,above, bend left] node {$p_{01}, a^{(1)}$} (s1);
      \path (s1) edge[sloped,midway,above] node {$p_{10}, a^{(1)}$} (s0);
      
      \path (s2) edge[sloped,midway,above, bend left] node {$p_{20}, a^{(2)}$} (s0);
      \path (s0) edge[sloped,midway,below] node {$p_{02}, a^{(2)}$} (s2);
      
      
      \path (s1) edge[sloped,midway,above, bend left] node {$p_{13}, a^{(2)}$} (s3);
      \path (s3) edge[sloped,midway,above] node {$p_{31}, a^{(1)}$} (s1);
      %\path (s1) edge[sloped,midway,above, bend left= 320] node {$p_{12}, a^{(3)}$} (s2);
      
      \path (s2) edge[sloped,midway,below] node {$p_{23}, a^{(1)}$} (s3);
      \path (s3) edge[sloped,midway,below,bend left] node {$p_{32}, a^{(1)}$} (s2);
      
      \path (s0) edge[sloped,midway,above, bend left] node {$p_{03}, a^{(3)}$} (s3);
      \path (s3) edge[sloped,midway,below, bend left] node {$p_{30}, a^{(3)}$} (s0);
      
      \path (s2) edge[sloped,midway,above,dashed, bend left] node {$p_{21}, a^{(3)}$} (s1);
      \path (s1) edge[sloped,midway,below,dashed, bend left] node {$p_{12}, a^{(3)}$} (s2);
      
      
      %\path (s2) edge[sloped,midway,above, bend below of =s3] node {$p_{21}, a^{(3)}$} (s1);
      
      \path (s0) edge[loop left, dashed] node {$p_{00}, a^{(0)} $} (s0);
      \path (s1) edge[loop right, dashed] node {$p_{00}, a^{(0)}$} (s1);
      \path (s2) edge[loop left, dashed] node {$p_{00}, a^{(0)}$} (s2);
      \path (s3) edge[loop right, dashed] node {$p_{00}, a^{(0)}$} (s3);
      
      %\path (s1) edge[loop right, dashed] node {$p_{11}$} (s1);
      \end{tikzpicture}
        \includegraphics[scale=0.2]{images/intersection2.png}
        \caption{Стохастический граф сети из двух двухфазных светофоров.}
        \label{fig:stohgraph3}
      \end{figure}
    \end{frame}
%%%%%%%%%%%%%%%%%%%%%%%%%%%%%%%%%%%%%%%%%%%%%%%%%%%%%%%%%
\begin{frame}
\justifying
\frametitle{Определения и обозначения}
    
    
    Функция суммарных доходов/вознаграждений агента при выбранной стратегии $\delta$  примет вид 
    \begin{equation}\label{VALUE}
        V\Bigl(\{S_t, \delta\}\Bigr) = \sum_{t=0}^{\infty}\gamma^t r(s_t,a_t),
    \end{equation}
        где $0 \leq \gamma \leq 1$ --- коэффициент переоценки, $\{S_t\}$ --- реализация случайного процесса $\xi(t)$ при выбранной стратегии $\delta =\{a_t, 0\leq t<\infty\}$.
    
\end{frame}

%%%%%%%%%%%%%%%%%%%%%%%%%%%%%%%%%%%%%%%%%%%%%%%%%%%%%%%%%
\begin{frame}
\justifying
\frametitle{Задача MARL для одного светофора}
    \begin{block}{Постановка задачи MARL для одного светофора}\justifying
        Требуется найти такое управление $\delta$, которое доставляет максимум функции 
        $$
        V\Bigl(\{S_t, \delta\}\Bigr),
        $$
        где функция на каждом шаге $t$ может быть определена для $s = S_t$
        \begin{equation}
            V^*(s)=\max\limits_a Q(s, a), 
        \end{equation}
        а $Q(s, a)$, есть функция
        \begin{equation}
            Q(s,a)  =  \sum_{s'\in S} p(s,a;s')\bigl(r(s,a;s')+\gamma \max_{a \in A}Q(s',a') \bigr).
        \end{equation}
        \end{block}

    
	\end{frame}

%%%%%%%%%%%%%%%%%%%%%%%%%%%%%%%%%%
\begin{frame}
\frametitle{Критерий сходимости}
\justifying

    Идея $Q$-обучения заключается в оценке невычислимой правой части:
    \begin{equation} \label{Qiteration}
        Q_{t+1}(s, a) =  Q_t(s, a) + \alpha _t (s, a) \left(r(s, a) +  \gamma \max_{a'\in A}  Q_t(s', a') - Q_t(s, a)\right)
    \end{equation}
    где $s'$ --- положение процесса на шаге $t+1$, если на шаге $t$ процесс был в состоянии $s$ и было выбрано действие $a$,
    $\alpha$ --- коэффициент скидки.

    Если на шаге $t$ процесс находился в состоянии $s$ и было выбрано действие $a$, то $0 \leq \alpha _t(s, a) \leq 1$, иначе $\alpha _t(s, a) = 0$.


\end{frame}

%%%%%%%%%%%%%%%%%%%%%%%%%%%%%%%%%%
\begin{frame}
\frametitle{Критерий сходимости}
\justifying
    
	$Q = \{ Q(s, a)\}_{s \in S, a' \in A}$ , можно записать итеративно $Q_{t+1} = A(Q_t)$, 
    
    где $A \colon \mathbb{R} _{\infty}^1 \to  \mathbb{R} _{\infty}^1$ --- сжимающее отображение.
    \begin{multline*}
    \rho ((A \circ Q_1)(s, a), (A \circ Q_2)(s, a)) % = \\
     %   = \max_{a'\in A, s'\in S}| \sum_{s'\in S}  p(s, a; s')(r(s, a) + \gamma \max_{a'\in A}  Q_1(s', a')) - \\
   %  - \sum_{s'\in S}  p(s, a; s')(r(s, a) + \gamma \max_{a'\in A}  Q_2(s', a'))|  
    \leq  \max |\gamma \max_{a'\in A}  Q_1(s', a')) - \gamma \max_{a'\in A}  Q_2(s', a'))| = \\
        = 	\gamma \rho (Q_1(s, a), Q_2(s, a)) ,  \gamma \in (0; 1)
    \end{multline*}

    \begin{block}{Критерий сходимости задачи}
        Если стратегия $a(\cdot )$ приводит к тому, что с вероятностью 1 каждая пара $(s, a)$ бесконечное число раз встречается,
         то из условия сжимаемости при
    \begin{equation}
        \sum _{t=0} ^\infty {\alpha _t (s, a)} = \infty, \qquad
        \sum _{t=0} ^\infty {\alpha _t (s, a)^2} \leq \infty
    \end{equation}
    cледует сходимость процесса (\ref{Qiteration})
    \end{block}
	

\end{frame}

%%%%%%%%%%%%%%%%%%%%%%%%%%%%%%%%%%

\begin{frame}
\frametitle{Критерий сходимости}
\justifying

    Решение задачи MARL для одного светофора имеет вид:
    \begin{equation}
	    V ^*(s) = \max_{a\in A}  \lim _{t \to + \infty}Q_t(s, a).
    \end{equation}
    \begin{equation}
	    a_t(s) = \arg  \max_{a' \in A} Q_t(s, a')
    \end{equation}	
    
\end{frame}

%%%%%%%%%%%%%%%%%%%%%%%%%%%%%%%%%%%%%%%%%%%%%%%%%%%%%%%%%
\begin{frame}
\justifying
\frametitle{Задача MARL для сети светофоров}

    
Функция суммарных доходов/вознаграждений для $K$ агентов примет вид 
\begin{equation}\label{VALUE}
    V\Bigl(\{S_t, \delta\}\Bigr) = \sum_{t=0}^{\infty}\gamma^t r(\textbf{s}_t,\textbf{a}_t),
\end{equation}
    где $0 \leq \gamma \leq 1$ --- коэффициент переоценки, $\{S_t\}$ --- реализация случайного процесса $\xi(t)$ при выбранной стратегии $\delta =\{\textbf{a}_t, 0\leq t<\infty\}$.




    
\end{frame}

%%%%%%%%%%%%%%%%%%%%%%%%%%%%%%%%%%%%%%%%%%%%%%%%%%%%%%%%%
\begin{frame}
\justifying
\frametitle{Задача MARL для сети светофоров}
\begin{block}{Постановка задачи MARL для сети светофоров}\justifying
    Требуется найти такое управление $\delta$, которое доставляет максимум функции 
    $$
    V\Bigl(\{S_t, \delta\}\Bigr),
    $$
    где функция на каждом шаге $t$ может быть определена для $\textbf{s} = S_t$
    \begin{equation}
        V^*( \textbf{s})=\max\limits_{\textbf{a}} Q(\textbf{s}, \textbf{a}), 
    \end{equation}
    а $Q(\textbf{s}, \textbf{a})$, есть функция
    \begin{equation}
        Q(\textbf{s}, \textbf{a})  =  \sum_{\textbf{s}'\in S} p(\textbf{s},\textbf{a};\textbf{s}')\bigl(r(\textbf{s},\textbf{a};\textbf{s}')+\gamma \max_{\textbf{a} \in A}Q(\textbf{s}',\textbf{a}') \bigr).
    \end{equation}
    \end{block}
    
\end{frame}

%%%%%%%%%%%%%%%%%%%%%%%%%%%%%%%%%%%%%%%%%%%%%%%%%%%%%%%%%
\begin{frame}
\justifying
\frametitle{Решение задачи MARL для двух светофоров}
    Рассмотрим агента $k$ для него определено множество соседних агентов $N(k)$
    Для системы из двух светофоров $N(k) = \{ j\}$
    Искать решение в таком виде неудобно, поэтому перепишем функцию $Q(\textbf{s},\textbf{a})$ как функцию обучения одного агента $k$, т.е. $Q^k(\textbf{s},a)$: 
    \begin{equation}
        Q^{k}(\textbf{s},a_k) 
        = \sum\limits_{a_j \in A^j} p(\textbf{s},a_j)p(\textbf{s},a_k)  Q(\textbf{s},\textbf{a})
    \end{equation}
    \begin{equation}
        Q^{kj}(\textbf{s},\textbf{a}) 
        = p(\textbf{s},a^j)p(\textbf{s},a^k)  Q(\textbf{s},\textbf{a}^{kj})
    \end{equation}
    Находим максимизирующее решение для $Q$-процесса одного агента $k$ на шаге $t$
    \begin{equation}
        a_t^k(\textbf{s}) = \arg  \max\limits_{a_k \in A^k} \sum\limits_{a_j \in A^j} Q_t^{kj}(\textbf{s},\textbf{a}^{kj})
    \end{equation}
 
\end{frame}

%%%%%%%%%%%%%%%%%%%%%%%%%%%%%%%%%%%%%%%%%%%%%%%%%%%%%%%%%
\begin{frame}
\justifying
\frametitle{Решение задачи MARL для сети светофоров}
    Имея алгоритм решения задачи для 2ух светофоров, можно получить решение для любого их количества в сети. 
    
    Для этого используем алгоритм MARLIN \cite{BOOK1}.

    Оптимальное управление для фиксированного агента $k$ будем искать как решение задачи MARL в виде:

    $a_k = \arg\max_{a_k\in A^k} \sum\limits_{j\in N(k)} \sum\limits_{a_j \in A^j} Q^{kj}_t( \textbf{s}, \textbf{a}^{kj}) p(\textbf{s}'| (\textbf{s},\textbf{a}^{kj}))$.
\end{frame}

%%%%%%%%%%%%%%%%%%%%%%%%%%%%%%%%%%

% \begin{frame}
% \frametitle{Алгоритм}
% \justifying

% \begin{figure}[h!]
%     \centering
        
%         \includegraphics[scale = 0.5]{images/blol.png}
%         \label{fig:block}
%         \caption{4. Блок-схема алгоритма}
% \end{figure}

% \end{frame}
%%%%%%%%%%%%%%%%%%%%%%%%%%%%%%%%%%
\begin{frame}
    \frametitle{Вычислительные эксперименты}
    \justifying
    
    \begin{block}{Цель}\justifying
        Сравнить время задержки машин в модели системы управления светофоров, длительность фаз которой получена перебором,
        и управляемой марковским процессом, в системе Anylogic.
    \end{block}

    \begin{block}{Входные данные} \justifying  
        \begin{enumerate}[-] \justifying
            \item машины прибывают на перекресток с каждого из трех направлений с интенсивностью 1000 в час;
            \item коэфициенты скидки и переоценки подобраны эмпирическим путём.
        \end{enumerate}
    \end{block}

    \begin{block}{Результаты и выводы} \justifying  
        \begin{enumerate}[-] \justifying
            \item ускорение в 1.5 раза по сравнению с системой управления светофором, длительность фаз которой подобрана перебором от 5 секунд до 30 секунд с шагом 1;
        \end{enumerate}
    \end{block}


\end{frame}
    
%%%%%%%%%%%%%%%%%%%%%%%%%%%%%%%%%%
\begin{frame}
\frametitle{Основные результаты работы}
\justifying
Целью работы являлось ознакомление с подходом, позволяющим оптимизировать процесс выбора сигнала светофора, с учетом текущей загрузки транспорта, с точки зрения минимизации задержки, 
а также создание алгоритма, реализующего данный подход   и вычисление задержек с его пмощью.
В работе получены следующие результаты:

\begin{enumerate}
	\item Построена математическая модель процесса выбора фазы светофора, отличающаяся учетом текущего расположения светофоров и их загрузки и позволяющая сформулировать оптимизационные задачи, целью которых является минимизация задержки трафика автомобилей.
	\item Разработана структура мультиагентной системы, включающая в себя единственного агента – светофор, обеспечивающая наиболее эффективное распараллеливание всей задачи на подзадачи, которые будут решены агентом.
\end{enumerate}	


\end{frame}

%%%%%%%%%%%%%%%%%%%%%%%%%%%%%%%%%%
\begin{frame}{Основная литература}\justifying
    \begin{thebibliography}{9}
        \bibitem{BOOK1}\label{IEE}
            El-Tantawy S., Abdulhai B. and Abdelgawad H., Multiagent Reinforcement Learning for Integrated Network of Adaptive Traffic Signal Controllers (MARLIN-ATSC) // IEEE Transactions on Intelligent Transportation Systems, vol. 14, no. 3. 2013. -P 1140-1150.
        \bibitem{BOOK2}
            Лекции по случайным процессам: учебное пособие / А. В. Гасников, Э. А. Горбунов, С. А. Гуз и др.; под ред. А. В. Гасникова. – «Москва»: МФТИ, 2019. – 285 с. 
        \bibitem{BOOK3}
            Марковские процессы принятия решений. / Майн X., Осаки С. Главная редакция физико-математической литературы издательства «Наука», 1977. – 176 с. 
        \bibitem{BOOK4}
            Sandholm T.W. Contract Types for Satisficing Task Allocation: I Theoretical Results // AAAI Spring Symposium Series: Satisficing Models. 1998. – P. 68-75.
        %
    \end{thebibliography}
\end{frame}
%%%%%%%%%%%%%%%%%%%%%%%%%%%%%%%%%%%%%%%%%%%%%%%%%%%%%%55
\begin{frame}
\begin{center}

{\color{blue}{\Huge{\bf СПАСИБО ЗА ВНИМАНИЕ!!!}}}
\end{center}
\end{frame}



\end{document}

