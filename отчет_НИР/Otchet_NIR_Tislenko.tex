\documentclass[10pt, mathserif, professionalfonts, aspectratio=1610]{beamer}

\usepackage[utf8]{inputenc}
\usepackage[english, russian]{babel}
\usepackage[T1, T2A]{fontenc}

\usepackage{pdfcomment}
\usepackage[inactive,blur=0.6, fixcolor]{fancytooltips}
\usepackage{pgfplotstable, pgf}
\usepackage{pgfplots}
\usepgfplotslibrary{patchplots}
\usepgfplotslibrary{fillbetween}

\usepackage{mathtools}
\usepackage{tikz-cd}
\usetikzlibrary{decorations.pathmorphing}

\usepackage{amsmath,amstext,amsfonts,amssymb,amsthm, mathrsfs}
\usepackage{mathtools}
\usepackage{cases}
\usepackage{ifthen}
\usepackage{array}
\usepackage{cellspace}
\usepackage{tabularx}
\usepackage{graphics}
\usepackage{graphicx}
\usepackage{picture}
\usepackage{ragged2e}
\usepackage{indentfirst}
\usepackage{hhline, slashbox}
\usepackage{hyperref}
\usepackage{pifont}
\usepackage{smartdiagram}
\usepackage{subcaption}

\usepackage{xcolor, colortbl}
\definecolor{LightCyan}{rgb}{0.88,1,1}
\definecolor{Gray}{gray}{0.9}
\definecolor{lightRed}{RGB}{250,225,200}

%\usepackage{color,colortbl}
\usepackage{tikz}
\usetikzlibrary{calc}
\usetikzlibrary{er}
\usetikzlibrary{automata}
\usetikzlibrary{graphs}
\usetikzlibrary{intersections}
\usetikzlibrary{positioning}
\usetikzlibrary{arrows}
\usetikzlibrary{arrows.meta}
\usetikzlibrary{spy}
\usepackage{pgf}
\usepackage{pgfplots}
\usepackage{rotating}
\usepackage{setspace}
\usepackage{multirow}
\usepackage{slashbox}

\usepackage{hhline, slashbox}
\usepackage[mathscr]{eucal}
\usepackage[ruled]{algorithm}
%\usepackage{algorithmic}
\usepackage{algorithmicx}
\usepackage{algpseudocode}
\usepackage{flafter}
\usepackage{caption}

\usepackage{enumitem}


\usetikzlibrary{arrows,shapes,positioning,shadows,trees}
\usetikzlibrary{calc}
\usetikzlibrary{decorations}

\makeatletter
\def\No{\textnumero}
\usepackage{pscyr}
\renewcommand{\rmdefault}{ftm} % Times New Roman
\renewcommand{\raggedright}{\leftskip=0.5pt \rightskip=0.5pt plus 0cm}



\usepackage[ruled]{algorithm}
\usepackage{ifthen}
\usepackage{algpseudocode}
\usepackage{enumitem}

\def\No{\textnumero} 
\def\labelenumi{\theenumi.}
\def\labelitemi{\textcolor{orange!90!black}{$\bullet$}}
%%%%%%%%%%%%%%%%%%%%%%%%%%%%%%%%%%%%%%%%%%%%%%%%%%%%%%%%%%%55



\definecolor{sfugrey}{RGB}{175,185,195}
\definecolor{sfugreydark}{RGB}{80,90,110}
\definecolor{sfuorange}{RGB}{230,110,36}

\usetheme{Madrid}
%%%%%%%%%%%%%%%%%%%%%%%%%%%%%%%%%%%%%%%%%%%%%%%%%%%%%%%%%%%%%%%%%%%5
% Русский псевдокод

\captionsetup[ruled]{labelsep=period}
\renewcommand{\thealgorithm}{\arabic{algorithm}}%

\floatname{algorithm}{Алгоритм}
\algrenewcommand\algorithmicrequire{\textbf{Вход: }}
\algrenewcommand\algorithmicensure{\textbf{Выход: }}
\algrenewcommand\algorithmicwhile{\textbf{До тех пор пока}}
	\algrenewcommand\algorithmicdo{\textbf{выполнять}}
	\algrenewcommand\algorithmicrepeat{\textbf{Повторять}}
	\algrenewcommand\algorithmicuntil{\textbf{Пока выполняется}}
	\algrenewcommand\algorithmicend{\textbf{Конец}}
	\algrenewcommand\algorithmicif{\textbf{Если}}
	\algrenewcommand\algorithmicelse{\textbf{иначе}}
	\algrenewcommand\algorithmicthen{\textbf{тогда}}
	\algrenewcommand\algorithmicfor{\textbf{Для}}
	\algrenewcommand\algorithmicforall{\textbf{Выполнить для всех}}
	\algrenewcommand\algorithmicfunction{\textbf{Функция}}
	\algrenewcommand\algorithmicprocedure{\textbf{Процедура}}
	\algrenewcommand\algorithmicloop{\textbf{Зациклить}}
	\algrenewcommand\algorithmicreturn{\textbf{Возвратить}}
	\algrenewtext{EndWhile}{\textbf{Конец цикла}}
	\algrenewtext{EndLoop}{\textbf{Конец зацикливания}}
	\algrenewtext{EndFor}{\textbf{Конец цикла}}
	\algrenewtext{EndFunction}{\textbf{Конец функции}}
	\algrenewtext{EndProcedure}{\textbf{Конец процедуры}}
	\algrenewtext{EndIf}{\textbf{Конец условия}}
	\algrenewtext{EndFor}{\textbf{Конец цикла}}
	\algrenewtext{BeginAlgorithm}{\textbf{Начало алгоритма}}
	\algrenewtext{EndAlgorithm}{\textbf{Конец алгоритма}}
	\algrenewtext{BeginBlock}{\textbf{Начало блока. }}
	\algrenewtext{EndBlock}{\textbf{Конец блока}}
	\algrenewtext{ElsIf}{\textbf{иначе если }}  
		
\algloop{description}
	\algnewcommand\algorithmicdescription{\textbf{Описание алгоритма}}
\renewcommand{\listalgorithmname}{Список алгоритмов}
%\renewcommand{\@biblabel}[1]{#1.}
%%%%%%%%%%%%%%%%%%%%%%%%%%%%%%%%%%%%%%%%%%%%%%%%%%%%%%%%%%%%%%%%%%%%%%%%%%%%%%%55555
%%%%%%%%%%%%%%%%%%%%%%%%%%%%%%%%%%%%%%%%%%%%%%%%%%%%%%%%%%%%%%%%%%%%%%%%%%%%%%%%%%55

\makeatletter

\setbeamercolor*{structure}{bg=sfugrey, fg=black}

\setbeamercolor*{palette primary}{use=structure,fg=white,bg=sfugreydark}
\setbeamercolor*{palette secondary}{use=structure,fg=white,bg=sfugrey!75}
\setbeamercolor*{palette tertiary}{use=structure,fg=white,bg=sfugrey!50!black}
\setbeamercolor*{palette quaternary}{fg=white,bg=sfugrey!50!white}


\setbeamercolor*{sidebar}{use=structure,bg=structure.fg}
  
\setbeamercolor*{palette sidebar primary}{use=structure,fg=structure.fg!10}
\setbeamercolor*{palette sidebar secondary}{fg=white}
\setbeamercolor*{palette sidebar tertiary}{use=structure,fg=structure.fg!50}
\setbeamercolor*{palette sidebar quaternary}{fg=white}


% Logo to use in the alternative title page.
\def\beamer@fancy@titlepagelogo{}
\DeclareOptionBeamer{titlepagelogo}{\def\beamer@fancy@titlepagelogo{#1}}
% Colors.
%\setbeamercolor*{title page header}{parent=sfugreydark}
\setbeamercolor*{lineup}{parent=palette quaternary}
\setbeamercolor*{linemid}{parent=palette secondary}
\setbeamercolor*{linebottom}{parent=palette tertiary}
\setbeamercolor*{title page header}{parent=palette primary}

% Lengths.
\newlength{\beamer@fancy@lineup}
\setlength{\beamer@fancy@lineup}{.025\paperheight}
\newlength{\beamer@fancy@linemid}
\setlength{\beamer@fancy@linemid}{.015\paperheight}
\newlength{\beamer@fancy@linebottom}
\setlength{\beamer@fancy@linebottom}{.01\paperheight}

% Margins.
\newlength{\beamer@fancy@normalmargin}
\setlength{\beamer@fancy@normalmargin}{.06\paperwidth}
\setbeamersize{text margin left=\beamer@fancy@normalmargin}
\setbeamersize{text margin right=\beamer@fancy@normalmargin}
\setlength\leftmargini{.6\beamer@fancy@normalmargin}
\setlength\leftmarginii{.6\beamer@fancy@normalmargin}
\setlength\leftmarginiii{.6\beamer@fancy@normalmargin}

\setbeamertemplate{title page}{
  {\parskip0pt\offinterlineskip%
  \hbox{\hskip-\Gm@lmargin\hbox{\vbox{%
  \@tempdima=\textwidth\textwidth=\paperwidth\hsize=\textwidth\def\\{,}\vbox{}\vskip-1.5ex%
    % Title.
    \begin{beamercolorbox}[wd=\paperwidth,ht=.4\paperheight,center]{title page header}
      \usebeamerfont{title}\inserttitle\par%
      \ifx\insertsubtitle\@empty%
      \else%
        \vskip0.25em%
        {\usebeamerfont{subtitle}\usebeamercolor[fg]{subtitle}\insertsubtitle\par}%
      \fi%     
      \vspace{.125\paperheight}%
    \end{beamercolorbox}%
    \vbox{}\vskip-\beamer@fancy@lineup%
    \vbox{}\vskip-\beamer@fancy@linemid%
    % First line.
    \hbox{%
    \begin{beamercolorbox}[wd=.2\paperwidth,ht=\beamer@fancy@lineup,dp=0pt]{}%
    \end{beamercolorbox}%
    \begin{beamercolorbox}[wd=.8\paperwidth,ht=\beamer@fancy@lineup,dp=0pt]{lineup}%
    \end{beamercolorbox}%
    }%
    \vbox{}\vskip0ex%
    % Second line.
    \hbox{%
    \begin{beamercolorbox}[wd=.1\paperwidth,ht=\beamer@fancy@linemid,dp=0pt]{}%
    \end{beamercolorbox}%
    \begin{beamercolorbox}[wd=.9\paperwidth,ht=\beamer@fancy@linemid,dp=0pt]{linemid}%
    \end{beamercolorbox}%
    }%
    % Third line.
    \hbox{%
    \begin{beamercolorbox}[wd=.5\paperwidth,ht=\beamer@fancy@linebottom,dp=0pt]{}%
    \end{beamercolorbox}%
    \begin{beamercolorbox}[wd=.5\paperwidth,ht=\beamer@fancy@linebottom,dp=0pt]{linebottom}%
    \end{beamercolorbox}%
    }%
    \vskip0pt%
  }}%
  \hskip-\Gm@rmargin%
  }}\hfil%
  %
  \begin{columns}
          \column{.5\textwidth}
              % Logo.
              \begin{centering}
                  %\vbox{}\vfill
                  \includegraphics[height=.4\paperheight]{logo_vertical}
                  \vfill
              \end{centering}
          \column{.5\textwidth}
         \begin{centering}
                  %\vbox{}\vfill
                  \includegraphics[height=.3\paperheight]{logoIM}
                  \vfill
              \end{centering}
          \vskip1em\par
          \begin{beamercolorbox}[sep=8pt,center]{author}
            \usebeamerfont{author}\insertauthor
          \end{beamercolorbox}
          \begin{beamercolorbox}[sep=8pt,center]{institute}
            \usebeamerfont{institute}\insertinstitute
          \end{beamercolorbox}
          %\begin{beamercolorbox}[sep=8pt,center]{date}
           % \usebeamerfont{date}\insertdate
          %\end{beamercolorbox}\vskip0.5em
          %{\usebeamercolor[fg]{titlegraphic}\inserttitlegraphic\par}
  \end{columns}
}



%\logo{\includegraphics{logo_vertical}}
%%%%%%%%%%%%%%%%%%%%%%%%%%%%%%%%%%%%%%%%%%%%%%%%%%%%%%%%%%%%%%%%%%%%%%%%%%%%%%%%%%%%%%%5
\makeatother

\captionsetup{figurename=Рисунок}
\captionsetup{tablename=Таблица }


\usefonttheme[onlymath]{serif}
%\usetheme[height=0.8cm, numbers, totalnumbers]{Rochester}

\setbeamersize{text margin left=0.45cm, text margin right=0.45cm}



\usecolortheme[RGB={230,110,36}]{structure}

\setbeamercolor{footlinecolor}{fg=black,bg=sfugrey}
\setbeamercolor{footlinecolordark}{fg=white,bg=sfugreydark}

\setbeamercolor{caption name}{fg=sfugreydark}

\setbeamercolor{frametitle}{bg=sfugreydark}%sfuorange}
\setbeamercolor{frametitle}{fg=white}

\setbeamercolor{block title}{bg=sfugreydark, fg=white}
%\setbeamercolor{block body}{bg=sfugrey, fg=black}

%\setbeamercolor{alertblock title}{bg=sfuorange, fg=white}

\setbeamercolor{block title example}{bg=sfugreydark, fg=white}
\setbeamercolor{block body example}{bg=white!100!sfuorange,fg=sfugreydark!25!black}


\setbeamercolor{block title alerted}{bg=sfuorange, fg=white}
\setbeamercolor{block body alerted}{bg=white!90!sfuorange,fg=sfugreydark!25!black}


\setbeamerfont{institute}{size=\normalsize}
\setbeamerfont{author}{size=\large}
\setbeamerfont{date}{size=\footnotesize}

%\usecolortheme{seahorse}%%{beaver}%{dolphin}%{whale}%{wolverine}{beaver}{crane}
\usefonttheme[onlymath]{serif}
\usefonttheme{structurebold}%{serif}%{structuresmallcapsserif}%{structureitalicserif}%{professionalfonts}
\useinnertheme[shadow]{rounded}
\setbeamertemplate{footline}
{%
 \leavevmode%
    \hbox{%
		        \begin{beamercolorbox}[wd=.6\paperwidth,ht=3ex,dp=1.125ex,leftskip=.3cm,rightskip=.3cm plus1fil, center]{footlinecolor}%
        						 \usebeamerfont{title in head/foot}\insertshorttitle
 												\end{beamercolorbox}
        \begin{beamercolorbox}[wd=.4\paperwidth,ht=3ex,dp=1.125ex,leftskip=.3cm,rightskip=.3cm plus1fil]{footlinecolordark}%
            \usebeamerfont{title in head/foot}%\insertshorttitle
					 \insertshortauthor \hfill \insertframenumber{}/\inserttotalframenumber
        \end{beamercolorbox}}%
    \vskip0pt%
}
\setbeamertemplate{navigation symbols}{}
\setbeamertemplate{caption}[numbered]
\renewcommand\baselinestretch{1.1}
\setbeamertemplate{enumerate items}[\insertenumlabel]


% Items.
\defbeamertemplate{itemize item}{squarealt}%
{\tiny\raise.5ex\hbox{\donotcoloroutermaths$\blacksquare$}}
\defbeamertemplate{itemize subitem}{squarealt}%
{\tiny\raise.4ex\hbox{\donotcoloroutermaths$\square$}}
\defbeamertemplate{itemize subsubitem}{squarealt}%
{\tiny\raise.3ex\hbox{\donotcoloroutermaths$\blacksquare$}}

%\renewcommand{\labelitemi}{--}
\newcommand{\specialcell}[2][l]{%
  \begin{tabular}[#1]{@{}c@{}}#2\end{tabular}}

\newenvironment<>{varblock}[2][.9\textwidth]{%
  \setlength{\textwidth}{#1}
  \begin{actionenv}#3%
    \def\insertblocktitle{#2}%
    \par%
    \usebeamertemplate{block begin}}
  {\par%
    \usebeamertemplate{block end}%
  \end{actionenv}}


  \title[MARL  ДЛЯ СЕТИ СВЕТОФОРОВ]{ИТММ'21\qquad ЗАДАЧА MARL  ДЛЯ СЕТИ СВЕТОФОРОВ}
\author[Тисленко~Т.\,И., Семенова~Д.\,В.]{\large Тисленко~Т.\,И., Семенова~Д.\,В.}

\date{2021}

\newcommand\xrsquigarrow[1]{%
    \mathrel{%
        \begin{tikzpicture}[%
            baseline={(current bounding box)}
            ]
        \node[%
            ,inner sep=.44ex
            ,align=center,
            ] (tmp) {$\scriptstyle #1$};
        \path[%
            ,draw,<-
            ,decorate,decoration={%
                ,zigzag
                ,amplitude=0.7pt
                ,segment length=1.2mm,pre length=3.5pt
                }
            ] 
        (tmp.north east) -- (tmp.north west);
        \end{tikzpicture}
        }
    }
\newcommand\xlsquigarrow[1]{%
    \mathrel{%
        \begin{tikzpicture}[%
            ,baseline={(current bounding box.south)}
            ]
        \node[%
            ,inner sep=.44ex
            ,align=center
            ] (tmp) {$\scriptstyle #1$};
        \path[%
            ,draw,<-
            ,decorate,decoration={%
                ,zigzag
                ,amplitude=0.7pt
                ,segment length=1.2mm,pre length=3.5pt
                }
            ] 
        (tmp.south west) -- (tmp.south east);
        \end{tikzpicture}
        }
    }



%%%%%%%%%%%%%%%%%%%%%%%%%%%%%%%%%%%%%%%%%%%%
\begin{document}



%\frame{\titlepage}
%\frame{\tableofcontents[currentsubsection]}

%-1---------------------- Титульный --------------------------------------
%\usebackgroundtemplate{\includegraphics[width=\paperwidth]{2}}%
\begin{frame}[plain]

\begin{block}{
  \huge {Отчет по научно-исследовательской работе 

на тему: <<MARL  ДЛЯ СЕТИ СВЕТОФОРОВ>>
}
}
\end{block}

\vspace*{2cm}


\begin{center}
  \textbf{Тисленко Тимофей Иванович}
  \bigskip

  {\large ФГАОУ ВО «СИБИРСКИЙ ФЕДЕРАЛЬНЫЙ УНИВЕРСИТЕТ»
  
  Институт математики и фундаментальной информатики
  
  Кафедра высшей и прикладной математики}
  
  \bigskip
  
  Научный руководитель — к.ф.-м.н., доцент Д.В. Семенова
  
  \bigskip
  
  Красноярск, 2021
  
\end{center}


\end{frame}

%%%%%%%%%%%%%%%%%%%%%%%%%%%%%%%%%%

\begin{frame}
	\frametitle{Актуальность}
    \justifying
    \begin{block}{Проблема}\justifying
      оптимизации планов  координации светофорных объектов в условиях городской сети.
    \end{block}
				        \begin{figure}[h!]
            \includegraphics[height=0.6\textheight]{images/jams.png}
            \caption{Пробки в Красноярске}
          \end{figure}
\end{frame}
%%%%%%%%%%%%%%%%%%%%%%%%%%%%%%%%%%


\begin{frame}
    \frametitle{Цели и задачи}
    \justifying
    \begin{block}{Цель работы}
        \justifying
        Разработка и исследование математической модели мультиагентной системы для задачи координации светофорных объектов в транспортной сети мегаполиса.
    \end{block}

    %\pause

    \begin{block}{Задачи}
        \begin{enumerate} \justifying
            \item Построить математическую модель MARL для сети двухфазных светофоров.
            \item Разработать алгоритм решения задачи MARL.
            \item Провести вычислительные эксперименты для одного и двух светофоров.
        \end{enumerate}
    \end{block}

  \end{frame}

  %%%%%%%%%%%%%%%%%%%%%%%%%%%%%%%%%%%%%%%%%%%%%%%%%%%%%%%%%
\begin{frame}
  \justifying
  \frametitle{Основные определения и обозначения}
  \begin{table}
	  \caption{Основные обозначения}
    \begin{tabular}{||c||l|} \hline
      $K$                     & число агентов                \\\hline
      $k$                     & номер агента                \\\hline
      $S^k$                   & пространство состояний для $k$-го агента     \\\hline
      $A^k$                   & множество решений   для $k$-го агента        \\\hline

      $\textbf{s}_t$          & совокупное состояние среды в момент времени $t$                                           \\\hline
      $\textbf{a}_t$          & совокупное управление в момент времени $t$                                                   \\\hline

      $r(s_t, a_t; s_{t+1})$  & время, подсчитанное для машин двигающихся на фазе $s_{t+1}$                                     \\\hline

      $N(k)$                  & множество соседних с $k$ агентов                                               \\\hline
      $\textbf{s}^{kj}$       & совместные состояния агентов    $k$ и $j$                                            \\\hline
      $\textbf{a}^{kj}$       & совместные действия агентов $k$ и $j$                                             \\\hline
      $ V\Bigl(\{S_t, \delta\}\Bigr)$ & функция суммарных доходов/вознаграждений для $K$ агентов                                        \\\hline

    \end{tabular}
  \end{table}
  
\end{frame}

%%%%%%%%%%%%%%%%%%%%%%%%%%%%%%%%%%%%%%%%%%%%%%%%%%%%%%%%%
\begin{frame}[shrink]
\justifying
\frametitle{Основные определения и обозначения}

    \begin{block}{Для  $k$-го агента }
        \begin{itemize} \justifying    
            \item $S^k$ определяются классом вычетов по модулю $$S^k = n^{(k)}\mathbb{Z}  = \left\{s^{(0)} = 0, s^{(1)} = 1, \ldots, s^{\left(n^{(k)}-1\right)} = n^{(k)}-1\right\}, \quad n^{(k)}=|S^k|,$$  где  $s^{(i)}$ интерпретируется как <<активна фаза $i$>>;
            \item $A^k$ определяется характеристикой кольца $n^{(k)}\mathbb{Z}$, т.е. не превыщает ее,\linebreak
            $A^k =\left\{a^{(0)} = 0, a^{(1)} = 1, \ldots, a^{(n^{(k)}-1)} = n^{(k)}-1 \right\}$, где $a^{(j)}$ интерпретируется как       <<сменить фазу $j$ раз>>. 
        \end{itemize}
    \end{block}

      \begin{block}{В момент времени $t$}
        \begin{itemize} \justifying    
            \item  $\textbf{s}_t = \{s_t^1, s_t^2, \ldots, s_t^K\}\in {\cal S}=S^1\times S^2\times\ldots\times S^K$;
            \item $\textbf{a}_t = \{a_t^1, a_t^2, \ldots, a_t^K\} \in  {\cal A}=A^1\times A^2\times\ldots\times A^K$; 
            \item $\textbf{a}^{kj} \in A^k\times A^j$ и $\textbf{s}^{kj} \in S^k\times S^j$;   
            \item $\displaystyle V\Bigl(\{\textbf{s}_t, \delta\}\Bigr) = \sum\limits_{t=0}^{\infty}\gamma^t r(\textbf{s}_t,\textbf{a}_t)$, где $0 \leq \gamma \leq 1$ --- коэффициент переоценки, $\{\textbf{s}_t\}$ --- реализация случайного процесса %$\xi(t)$ 
            при выбранной стратегии $\delta =\{\textbf{a}_t, 0\leq t<\infty\}$.   
        \end{itemize}
    \end{block}


  
\end{frame}

%%%%%%%%%%%%%%%%%%%%%%%%%%%%%%%%%%%%%%%%%%%%%%%%%%%%%%%%%
\begin{frame}
\justifying
\frametitle{Постановка задачи MARL}


\begin{figure}[h!]
    %\centering
    \begin{tikzpicture}[->,>=stealth',shorten >=1pt,auto,node distance=3cm,
    semithick]
\tikzstyle{every state}=[draw,text=black]

\node[state] (A)  {$s^{(0)}$};
\node[state] (B) [right of=A] {$s^{(1)}$};


\path (A) edge[bend right] node {$p_{01}$} (B);
\path (B) edge[bend right] node {$p_{10}$} (A);
\path (A) edge[loop left, dashed] node {$p_{00}$} (A);
\path (B) edge[loop right, dashed] node {$p_{11}$} (B);
\end{tikzpicture}
\includegraphics[width =0.3\textwidth]{images/intersection1.png}
    \caption{Стохастический граф управляемого процесса смены фаз для двухфазного светофора и его визуальная интерпретация.  Пунктиром обозначено действие $a^{(0)}$, а сплошной --- $a^{(1)}$}
    \label{fig:stohgraph}
\end{figure}


\end{frame}
%%%%%%%%%%%%%%%%%%%%%%%%%%%%%%%%%%%%%%%%%%%%%%%%%%%%%%%%%
\begin{frame}
    \justifying
    \frametitle{Постановка задачи}
    
    
    \begin{figure}[h!]
        %\centering
        %\includegraphics[width =0.2\textwidth]{images/intersection2.png}
        \begin{tikzpicture}[->,>=stealth',shorten >=1pt,auto,node distance=3.8cm,
          semithick, scale=0.1]
      \tikzstyle{every state}=[draw,text=black]
      
      \node[state] (s0)  {$s_{0}$};
      \node[state] (s1) [right of=s0] {$s_{1}$};
      \node[state] (s2) [below  of=s0] {$s_{2}$};
      \node[state] (s3) [below  of=s1] {$s_{3}$};
      
      \path (s0) edge[sloped,midway,above, bend left] node {$p_{01}, a^{(1)}$} (s1);
      \path (s1) edge[sloped,midway,above] node {$p_{10}, a^{(1)}$} (s0);
      
      \path (s2) edge[sloped,midway,above, bend left] node {$p_{20}, a^{(2)}$} (s0);
      \path (s0) edge[sloped,midway,below] node {$p_{02}, a^{(2)}$} (s2);
      
      
      \path (s1) edge[sloped,midway,above, bend left] node {$p_{13}, a^{(2)}$} (s3);
      \path (s3) edge[sloped,midway,above] node {$p_{31}, a^{(1)}$} (s1);
      %\path (s1) edge[sloped,midway,above, bend left= 320] node {$p_{12}, a^{(3)}$} (s2);
      
      \path (s2) edge[sloped,midway,below] node {$p_{23}, a^{(1)}$} (s3);
      \path (s3) edge[sloped,midway,below,bend left] node {$p_{32}, a^{(1)}$} (s2);
      
      \path (s0) edge[sloped,midway,above, bend left] node {$p_{03}, a^{(3)}$} (s3);
      \path (s3) edge[sloped,midway,below, bend left] node {$p_{30}, a^{(3)}$} (s0);
      
      \path (s2) edge[sloped,midway,above,dashed, bend left] node {$p_{21}, a^{(3)}$} (s1);
      \path (s1) edge[sloped,midway,below,dashed, bend left] node {$p_{12}, a^{(3)}$} (s2);
      
      
      %\path (s2) edge[sloped,midway,above, bend below of =s3] node {$p_{21}, a^{(3)}$} (s1);
      
      \path (s0) edge[loop left, dashed] node {$p_{00}, a^{(0)} $} (s0);
      \path (s1) edge[loop right, dashed] node {$p_{00}, a^{(0)}$} (s1);
      \path (s2) edge[loop left, dashed] node {$p_{00}, a^{(0)}$} (s2);
      \path (s3) edge[loop right, dashed] node {$p_{00}, a^{(0)}$} (s3);
      
      %\path (s1) edge[loop right, dashed] node {$p_{11}$} (s1);
      \end{tikzpicture}
        \includegraphics[scale=0.2]{images/intersection2.png}
        \caption{Стохастический граф сети из двух двухфазных светофоров}
        \label{fig:stohgraph3}
      \end{figure}
\end{frame}


%%%%%%%%%%%%%%%%%%%%%%%%%%%%%%%%%%%%%%%%%%%%%%%%%%%%%%%%%
\begin{frame}
\justifying
\frametitle{Задача MARL для сети светофоров}
    \begin{block}{Постановка задачи MARL для сети светофоров}\justifying
        Требуется найти такое управление $\delta=\{\textbf{a}_t, 0\leq t<\infty\}$, которое доставляет максимум функции суммарных вознаграждений для $K$ агентов 
        $$
\displaystyle V\Bigl(\{\textbf{s}_t, \delta\}\Bigr) = \sum\limits_{t=0}^{\infty}\gamma^t r(\textbf{s}_t,\textbf{a}_t) \rightarrow \max,$$ где $0 \leq \gamma \leq 1$ --- коэффициент переоценки, $\{\textbf{s}_t\}$ --- реализация случайного процесса.
            \end{block}
   Решение ищется методом динамического программирования на основе принципа оптимальности Беллмана.
   Функцию суммарных вознаграждений при оптимальном управлении на шаге $t$: 
    \begin{equation}
        V^*(\{\textbf{s}_{t'}\}_{t'=0}^{t'=t})=\max\limits_{\textbf{a}\in{\cal A}} Q_t(\textbf{s}_t, \textbf{a}), 
    \end{equation}
где 
    \begin{equation}
        Q_t(\textbf{s}_t, \textbf{a})  =    \sum\limits_{\textbf{s}_{t+1}\in {\cal S}} p(\textbf{s}_t,\textbf{a};\textbf{s}_{t+1})\Bigl(r(\textbf{s}_t,\textbf{a};\textbf{s}_{t+1})+\gamma \max\limits_{\textbf{a}' \in {\cal A}}Q_{t-1}(\textbf{s}_{t+1},\textbf{a}') \Bigr).
    \end{equation}
\end{frame}

%%%%%%%%%%%%%%%%%%%%%%%%%%%%%%%%%%%%%%%%%%%%%%%%%%%%%%%%%
\begin{frame}
\justifying
\frametitle{Решение задачи MARL для двух светофоров}
   
	 \begin{itemize}
		 \item Множество соседних агентов $N(k) = \{j\}$, $k\ne j$, $k,j =0, 1$.
		\item Функция обучения  агента $k$
		    \begin{equation}
        Q_t^{k}(\textbf{s}_t,a^k_t)= \sum\limits_{a_j \in A^j} \underbrace{p(a^k_t;\textbf{a}^{kj}_t)Q_t(\textbf{s}_t,\textbf{a}^{kj}_t)}_{  Q^{kj}_t(\textbf{s}_t,\textbf{a}^{kj}_t) }
				= \sum\limits_{a^j \in A^j} Q_t^{kj}(\textbf{s},\textbf{a}^{kj}_t).
    \end{equation}
	 \end{itemize}
  \begin{multline}
    Q_t^{k}(\textbf{s}_t,a^k_t)= \sum\limits_{a_j \in A^j} p(a^k_t;\textbf{a}^{kj}_t) \left( \sum\limits_{\textbf{s}_{t+1}\in {\cal S}} p(\textbf{s}_t,\textbf{a}^{kj}_t;\textbf{s}_{t+1})\Bigl(r(\textbf{s}_t,\textbf{a}^{kj}_t;\textbf{s}_{t+1})+\gamma \max\limits_{\textbf{a}' \in {\cal A}}Q_{t-1}(\textbf{s}_{t+1},\textbf{a}') \Bigr)\right) = \\
    = \sum\limits_{\textbf{s}_{t+1}\in {\cal S}} \sum\limits_{a_j \in A^j}  p(\textbf{s}_t,a^k_t;\textbf{s}_{t+1}) \Bigl( r(\textbf{s}_t,\textbf{a}^{kj}_t;\textbf{s}_{t+1}) +\gamma \max\limits_{\textbf{a}' \in {\cal A}}Q_{t-1}(\textbf{s}_{t+1},\textbf{a}')\Bigr).
  \end{multline}
\begin{block}{Оптимальное решение для агента $k$ на шаге $t$}
    \begin{equation}
        a_t^k = \arg  \max\limits_{a^k \in A^k} \sum\limits_{a^j \in A^j} Q_t^{kj}(\textbf{s},\textbf{a}^{kj}).
    \end{equation}
 \end{block}
\end{frame}

%%%%%%%%%%%%%%%%%%%%%%%%%%%%%%%%%%%%%%%%%%%%%%%%%%%%%%%%%
\begin{frame}
\justifying
\frametitle{Решение задачи MARL для сети светофоров}
\begin{alertblock}{}
    Имея алгоритм решения задачи для двух светофоров, можно получить решение для любого их количества в сети. 
  \end{alertblock}  
    
		\begin{block}{Алгоритм MARLIN: основная идея}
    Оптимальное управление для фиксированного агента $k$ будем искать как решение задачи MARL в виде:
   \begin{equation} 
	a_t^k = \arg\max_{a^k \in A^k} \sum\limits_{j\in N(k)} \sum\limits_{a^j \in A^j} Q^{kj}_t( \textbf{s}, \textbf{a}^{kj}) p(\textbf{s}'| (\textbf{s},\textbf{a}^{kj}))
	   \end{equation}
		\end{block}
		
		\footnotesize{El-Tantawy S., Abdulhai B. and Abdelgawad H., Multiagent Reinforcement Learning for Integrated Network of Adaptive Traffic Signal Controllers (MARLIN-ATSC) // IEEE Transactions on Intelligent Transportation Systems, vol. 14, no. 3. 2013. --- P. 1140-1150.}
\end{frame}

%%%%%%%%%%%%%%%%%%%%%%%%%%%%%%%%%%

% \begin{frame}
% \frametitle{Алгоритм}
% \justifying

% \begin{figure}[h!]
%     \centering
        
%         \includegraphics[scale = 0.5]{images/blol.png}
%         \label{fig:block}
%         \caption{4. Блок-схема алгоритма}
% \end{figure}

% \end{frame}
%%%%%%%%%%%%%%%%%%%%%%%%%%%%%%%%%%
\begin{frame}
    \frametitle{Вычислительные эксперименты}
    \justifying
    Вычислительные эксперименты проводились в системе имитационного моделирования Anylogic. Алгоритмы реализованы на языке программирования Java 8. Построены модели перекрестка с фиксированной длительностью фаз (рис. 4а) и управляемого марковским процессом (рис. 4б).

    \begin{figure}[h!]
    
        \includegraphics[width =0.45\textwidth]{images/fixed.png}
        \includegraphics[width =0.4\textwidth]{images/vary.png}\label{fixed_vary}
				
				\hfill а) \hfill б) \hfill
				
        \caption{a) модель перекрестка c фиксированной длительностью фаз, б) модель перекрестка управляемого марковским процессом}
    \end{figure}

\end{frame}
    
%%%%%%%%%%%%%%%%%%%%%%%%%%%%%%%%%%
\begin{frame}
\frametitle{Основные результаты работы}
\justifying

\begin{figure}[h!]
    
    \includegraphics[width =1\textwidth]{images/7.png}
    \caption{Сравнение задержки $TIMESUMM$}
\end{figure}


\end{frame}
%%%%%%%%%%%%%%%%%%%%%%%%%%%%%%%%%%
\begin{frame}\frametitle{Основные результаты}
\justifying

\begin{itemize}
	\item Построена математическая модель процесса выбора фазы для сети светофоров, отличающаяся учетом текущего расположения светофоров и их загрузки и позволяющая сформулировать оптимизационные задачи, целью которых является минимизация задержки трафика автомобилей.
	\item Разработана структура мультиагентной системы --- сети сфетофоров участка дороги.
	\item Проведены вычислительные эксперименты  в системе имитационного моделирования Anylogic для одного и двух сфетофоров.
\end{itemize}	

\end{frame}
    
%%%%%%%%%%%%%%%%%%%%%%%%%%%%%%%%%%
\begin{frame}\frametitle{Аппробация работы}
  \justifying
    \begin{figure}[h!]
      
      
      \includegraphics[height = 0.3\textwidth]{images/dip2.png}
      \includegraphics[width =0.5\textwidth]{images/dip1.png}
      \caption{Дипломы конференции}

  \end{figure}

\end{frame}
    
%%%%%%%%%%%%%%%%%%%%%%%%%%%%%%%%%%
\begin{frame}\frametitle{Планы}
\justifying
\begin{figure}[h!]
    
    \includegraphics[width =0.5\textwidth]{images/set.png}
    \caption{Карта перекрестков, предоставленная УДиБ г. Красноярска}
\end{figure}


\end{frame}
    
%%%%%%%%%%%%%%%%%%%%%%%%%%%%%%%%%%
\begin{frame}\frametitle{Планы}
\justifying

\begin{figure}[h!]
    
    \includegraphics[width =0.5\textwidth]{images/Sv-G.png}
    \caption{Модель перекрестка проспект Свободный -- ул. Годенко}
\end{figure}

\end{frame}
%%%%%%%%%%%%%%%%%%%%%%%%%%%%%%%%%%
\begin{frame}\frametitle{Планы}
  \justifying
  
  \begin{block}{В следующем семестре планируется:}
    \begin{itemize}
      \item проверить эффективность модели на реальных объектах,
      \item разработать инструменты обучения модели, требующие меньших вычислительных ресурсов.
    \end{itemize}
  \end{block}
  \end{frame}
%%%%%%%%%%%%%%%%%%%%%%%%%%%%%%%%%%
\begin{frame}{Основная литература}\justifying
    \begin{thebibliography}{9}
        \bibitem{BOOK1}\label{IEE}
			El-Tantawy S., Abdulhai B. and Abdelgawad H., Multiagent Reinforcement Learning for Integrated Network of Adaptive Traffic Signal Controllers (MARLIN-ATSC) // IEEE Transactions on Intelligent Transportation Systems, vol. 14, no. 3. 2013. --- P. 1140-1150.
        \bibitem{BOOK2}
				Лекции по случайным процессам: учебное пособие / А. В. Гасников, Э. А. Горбунов, С. А. Гуз и др.; под ред. А. В. Гасникова. -- «Москва»: МФТИ, 2019. --- 285 с. 
        \bibitem{BOOK3}
            Марковские процессы принятия решений. / Майн X., Осаки С. Главная редакция физико-математической литературы издательства «Наука», 1977. --- 176 с. 
        \bibitem{BOOK4}
            Sandholm T.W. Contract Types for Satisficing Task Allocation: I Theoretical Results // AAAI Spring Symposium Series: Satisficing Models. 1998. --- P. 68-75.
        \bibitem{BOOK5}
        Управляемые марковские процессы с конечными пространствами состояний и управлений, Теория вероятностей и ее примен, Том 11. /В. В. Рыков. 1966. --- 343-351 с. 
        %
    \end{thebibliography}
\end{frame}

%%%%%%%%%%%%%%%%%%%%%%%%%%%%%%%%%%%%%%%

\end{document}

\begin{block}{Пример: дорожная ситуация (рис.2) }\justifying
    \footnotesize{
    $t = 0$, $s_0 = 0$ (для дороги 0).
    Коричневый седан на дороге 0 и едет 2с.  Красные машины на 1 стоят 5с. $r(s_0=0, a_0 = 0) = 2$ и $r(s_0=0, a_0 = 1) = 10$ --- вознаграждения.
    $Q_0(s_0,a) = (1 - \alpha)\cdot 0 + \alpha(r(s_0,a)+ \gamma \cdot 0). Q_0(0,0) = 2 \alpha, Q_0(0,1) = 5 \alpha.$
    Выбирается действие $a_0 = 1$.
    $t = 1$, $s_0 = 1$. $r(1, 0) = 0$ и $r(1, 1) = 7$
    $Q_1(1,0) = (1-\alpha)\cdot 10+ \alpha (0 + 10) =  10(1-\alpha) + 10\alpha\gamma,
    Q_1(1,1) = (1-\alpha)\cdot 5+ \alpha (5 + \gamma 10) =  2(1-\alpha) + 15\alpha\gamma$
    В работе подобраны коэффициенты $\alpha = 0.5, \gamma = 0.8$ 
    $Q_1(1,0) = 0.5 \cdot 5 + 0.5 \cdot 0.8 \cdot 5 = 4.5, Q_1(1,1) = 0.5 \cdot 2 + + 0.5 \cdot (7 + 0.8 \cdot 5 ) = 6.5$
    Выбирается действие $a_1 = 1$.
    }
\end{block}

\begin{frame}
  \justifying
  \frametitle{Задача MARL для одного светофора}
      \begin{block}{Постановка задачи MARL для одного светофора}\justifying
          Требуется найти такое управление $\delta$, которое доставляет максимум функции 
          $$
          V\Bigl(\{S_t, \delta\}\Bigr),
          $$
          где функция на каждом шаге $t$ может быть определена для $s = S_t$
          \begin{equation}
              V^*(s)=\max\limits_a Q(s, a), 
          \end{equation}
          а $Q(s, a)$, есть функция
          \begin{equation}
              Q(s,a)  =  \sum_{s'\in S} p(s,a;s')\bigl(r(s,a;s')+\gamma \max_{a \in A}Q(s',a') \bigr).
          \end{equation}
          \end{block}
  
      
    \end{frame}
  
  %%%%%%%%%%%%%%%%%%%%%%%%%%%%%%%%%%
  \begin{frame}
  \frametitle{Критерий сходимости}
  \justifying
  
      Идея $Q$-обучения заключается в оценке невычислимой правой части:
      \begin{equation} \label{Qiteration}
          Q_{t+1}(s, a) =  Q_t(s, a) + \alpha _t (s, a) \left(r(s, a) +  \gamma \max_{a'\in A}  Q_t(s', a') - Q_t(s, a)\right)
      \end{equation}
      где $s'$ --- положение процесса на шаге $t+1$, если на шаге $t$ процесс был в состоянии $s$ и было выбрано действие $a$,
      $\alpha$ --- коэффициент скидки.
  
      Если на шаге $t$ процесс находился в состоянии $s$ и было выбрано действие $a$, то $0 \leq \alpha _t(s, a) \leq 1$, иначе $\alpha _t(s, a) = 0$.
  
  
  \end{frame}
  
  %%%%%%%%%%%%%%%%%%%%%%%%%%%%%%%%%%
  \begin{frame}
  \frametitle{Критерий сходимости}
  \justifying
      
    $Q = \{ Q(s, a)\}_{s \in S, a' \in A}$ , можно записать итеративно $Q_{t+1} = A(Q_t)$, 
      
      где $A \colon \mathbb{R} _{\infty}^1 \to  \mathbb{R} _{\infty}^1$ --- сжимающее отображение.
      \begin{multline*}
      \rho ((A \circ Q_1)(s, a), (A \circ Q_2)(s, a)) % = \\
       %   = \max_{a'\in A, s'\in S}| \sum_{s'\in S}  p(s, a; s')(r(s, a) + \gamma \max_{a'\in A}  Q_1(s', a')) - \\
     %  - \sum_{s'\in S}  p(s, a; s')(r(s, a) + \gamma \max_{a'\in A}  Q_2(s', a'))|  
      \leq  \max |\gamma \max_{a'\in A}  Q_1(s', a')) - \gamma \max_{a'\in A}  Q_2(s', a'))| = \\
          = 	\gamma \rho (Q_1(s, a), Q_2(s, a)) ,  \gamma \in (0; 1)
      \end{multline*}
  
      \begin{block}{Критерий сходимости задачи}
          Если стратегия $a(\cdot )$ приводит к тому, что с вероятностью 1 каждая пара $(s, a)$ бесконечное число раз встречается,
           то из условия сжимаемости при
      \begin{equation}
          \sum _{t=0} ^\infty {\alpha _t (s, a)} = \infty, \qquad
          \sum _{t=0} ^\infty {\alpha _t (s, a)^2} \leq \infty
      \end{equation}
      cледует сходимость процесса (\ref{Qiteration})
      \end{block}
    
  
  \end{frame}
  
  %%%%%%%%%%%%%%%%%%%%%%%%%%%%%%%%%%
  
  \begin{frame}
  \frametitle{Критерий сходимости}
  \justifying
  
      Решение задачи MARL для одного светофора имеет вид:
      \begin{equation}
        V ^*(s) = \max_{a\in A}  \lim _{t \to + \infty}Q_t(s, a).
      \end{equation}
      \begin{equation}
        a_t(s) = \arg  \max_{a' \in A} Q_t(s, a')
      \end{equation}	
      
  \end{frame}
  

  
\end{document}

